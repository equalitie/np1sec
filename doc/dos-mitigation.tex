% Created 2016-03-11 Fri 23:07
\documentclass[11pt]{article}
\usepackage[utf8]{inputenc}
\usepackage[T1]{fontenc}
\usepackage{fixltx2e}
\usepackage{graphicx}
\usepackage{longtable}
\usepackage{float}
\usepackage{wrapfig}
\usepackage{rotating}
\usepackage[normalem]{ulem}
\usepackage{amsmath}
\usepackage{textcomp}
\usepackage{marvosym}
\usepackage{wasysym}
\usepackage{amssymb}
\usepackage{hyperref}
\tolerance=1000
\date{\today}
\title{DoS Mitigation in (n+1)sec protocol}
\hypersetup{
  pdfkeywords={},
  pdfsubject={},
  pdfcreator={Emacs 24.5.1 (Org mode 8.2.10)}}
\begin{document}

\maketitle
\tableofcontents

\section{DoS axioms:}
\label{sec-1}
\subsection{We assume that the joiner cannot deny access for other joiners because of the following:}
\label{sec-1-1}
\subsubsection{The line in the code which results in dropping other join requests:}
\label{sec-1-1-1}
   if (action$_{\text{to}}$$_{\text{take}}$.action$_{\text{type}}$ \texttt{= RoomAction::NEW\_PRIORITY\_SESSION ||
            action\_to\_take.action\_type =} RoomAction::PRESUME$_{\text{HEIR}}$) \{
    stale$_{\text{in}}$$_{\text{limbo}}$$_{\text{sessions}}$$_{\text{presume}}$$_{\text{heir}}$(action$_{\text{to}}$$_{\text{take}}$.bred$_{\text{session}}$->session$_{\text{id}}$);
\} // else \{ //user state in the room

\subsubsection{The condition that leads to PRESUME$_{\text{HEIR}}$}
\label{sec-1-1-2}
if (everybody$_{\text{authenticated}}$$_{\text{and}}$$_{\text{contributed}}$()) \{
    group$_{\text{dec}}$();
    // first compute the confirmation
    compute$_{\text{session}}$$_{\text{confirmation}}$();
    // we need our future ephemeral key to attach to the message
    future$_{\text{cryptic}}$.init();
    // now send the confirmation message
    Message outboundmessage(\&cryptic);

outboundmessage.create$_{\text{session}}$$_{\text{confirmation}}$$_{\text{msg}}$(
    session$_{\text{id}}$, hash$_{\text{to}}$$_{\text{string}}$$_{\text{buff}}$(session$_{\text{confirmation}}$),
    public$_{\text{key}}$$_{\text{to}}$$_{\text{stringbuff}}$(future$_{\text{cryptic}}$.get$_{\text{ephemeral}}$$_{\text{pub}}$$_{\text{key}}$()));

outboundmessage.send(room$_{\text{name}}$, us);

RoomAction re$_{\text{limbo}}$$_{\text{action}}$;

re$_{\text{limbo}}$$_{\text{action}}$.action$_{\text{type}}$ = RoomAction::PRESUME$_{\text{HEIR}}$;
re$_{\text{limbo}}$$_{\text{action}}$.bred$_{\text{session}}$ = this;

    return StateAndAction(GROUP$_{\text{KEY}}$$_{\text{GENERATED}}$, re$_{\text{limbo}}$$_{\text{action}}$);
    \emph{/ if we are joining we don't need to relimbo and the room will
    /} ignore the action
\}

\subsubsection{The condition that leads to NEW$_{\text{PRIORITY}}$$_{\text{SESSION}}$}
\label{sec-1-1-3}
RoomAction Session::shrink(std::string leaving$_{\text{nick}}$)
\{
 \emph{/ we are basically building the intention-to-leave message
 /} without broadcasting it (because it is not us who intend to do so),
 // so we are making a fake intention-to-leave message without sending it
 RoomAction new$_{\text{session}}$$_{\text{action}}$;

auto leaver = participants.find(leaving$_{\text{nick}}$);
if (leaver == participants.end()) \{
    logger.warn("participant " + leaving$_{\text{nick}}$ + " is not part of the active session of the room " + room$_{\text{name}}$ +
                " from which they are trying to leave, already parted?");
\} else if (zombies.find(leaving$_{\text{nick}}$) != zombies.end()) \{ // we haven't shrunk and made a session yet
    logger.debug("shrunk session for leaving user " + leaving$_{\text{nick}}$ + " has already been generated. nothing to do",
                 \uline{\uline{FUNCTION}}, myself.nickname);
\} else \{ \emph{/ shrink now
    /} if everything is ok, add the leaver to the zombie list and make a
    // session without zombies
    zombies.insert(*leaver);

// raison$_{\text{detre}}$.insert(RaisonDEtre(LEAVE, leaver->id));

Session* new$_{\text{child}}$$_{\text{session}}$ =
    new Session(PEER, us, room$_{\text{name}}$, \&future$_{\text{cryptic}}$, future$_{\text{participants}}$());

new$_{\text{session}}$$_{\text{action}}$.action$_{\text{type}}$ = RoomAction::NEW$_{\text{PRIORITY}}$$_{\text{SESSION}}$;
new$_{\text{session}}$$_{\text{action}}$.bred$_{\text{session}}$ = new$_{\text{child}}$$_{\text{session}}$;

    \emph{/ we are as we have farewelled
    /} my$_{\text{state}}$ = FAREWELLED; \emph{/We shouldn't change here and it is not clear why
    /} we need this stage: so as not to accept join? Why? Join will fail by non-confirmation
    // of the leavers
    return new$_{\text{session}}$$_{\text{action}}$;
\}

    return c$_{\text{no}}$$_{\text{room}}$$_{\text{action}}$;
\}

\subsubsection{The joiner DOSing}
\label{sec-1-1-4}
\begin{itemize}
\item As a result, the joiner cannot leave the session before joining, so she cannot
invoke NEW$_{\text{PRIORITY}}$$_{\text{SESSION}}$. 
\begin{itemize}
\item PRESUME$_{\text{HEIR}}$ is only called when group$_{\text{dec}}$ is successful: if the group dec is not correct, then the user will not drop the other sessions.
\item If the joiner sends a correct message to some users and a bad message to others, she can force some of the participants to drop the session and some not. Therefore, whenever a user is dropping a session, she should announce it if it is because of: 
\begin{itemize}
\item timeout
\item auth error
\item decryption error
\item confirmation error

\item If any user sends a decryption error or a confirmation error, then all users of the session run the cheater detection algorithm.
\end{itemize}

\item Badly signed messages are considered just garbage/not sent/not received.

\item Scenarios:
\begin{itemize}
\item A malicious joiner sends bad shares.
\item Results in decryption error. This still needs investigation because it could be some insider who did it. Other joiners join as they did before.
\item Some decrypt correctly and some decrypt badly. In this situation, we still run a DoS detection process. If the shares are all signed, then it is on the sender.
\end{itemize}

\item Dropping by timing:
\begin{itemize}
\item You request drop in the grace period and wait for the period to pass.
\item You accept drop if the grace period has passed.
\item If you received the messages of the party being dropped, you keep them. Drop the 
dropper.
\end{itemize}

\item Detection
\begin{enumerate}
\item If it fails due to session confirmation or decryption failure, the user should inform and request for re-session marked with DoS detection.
\item If it passes the second time, then you can go ahead, taking off the DoS tag.
\item If it fails again while tagged with DoS detection, everybody signs and sends their new private key with their old ephemeral public key and old established p2p keys.
\item Detect the cheater. Drop the cheater. Broadcast your proof.
\item If someone is dropping someone else without proper proof, drop them indicating lack of proof.
\end{enumerate}

\item Remedy:
\begin{enumerate}
\item The user drops the cheaters/DOSers from the new participant list (as they have indicated their intention to leave), and sends a new participant info message for the new message which tells other joiners to try again.
\item The user drops as many as she needs till she is alone.
\item When the user drops someone, she doesn't accept them as participants anymore, only as joiners.
\item Somebody wants to join as a participant but the user expects her as a joiner, so she informs her about that.
\end{enumerate}
\end{itemize}
\end{itemize}

\subsubsection{No joiner is receiving priority before}
\label{sec-1-1-5}
**

\subsection{If a set of participants cannot reach an agreement, the status quo will remain in place.}
\label{sec-1-2}
\subsection{Maliciousness is relative and defined based on the agreeing subgroup. If the transport delivers different payloads for different participants, then those sets of participants cannot reach an agreement.}
\label{sec-1-3}

\section{DoS goal:}
\label{sec-2}
\subsection{A set of benign participants where the transport honestly delivers packets between them should be able to form a session.}
\label{sec-2-1}

\section{List of DoS possibilities}
\label{sec-3}
\subsection{DoS Maliciousness}
\label{sec-3-1}
\begin{enumerate}
\item Unresponsiveness.
\item Generating wrong keyshare.
\item Confirming wrong session.
\item Asking for people to leave without reason.
\end{enumerate}
\subsection{Exception: Authentication discrepancy:}
\label{sec-3-2}
\subsubsection{If two participants do not agree on authenticating a new joiner, then the protocol halts without consequences because authentication is (1) deniable and (2) inherently a privilege.}
\label{sec-3-2-1}

\section{How to detect}
\label{sec-4}
\begin{itemize}
\item When a participant concludes that another participant is malicious
because of one of the above reasons, she request that participant to be 
kicked out of the participant list, and includes the reason for the kick.
\item How to detect DoS:
\begin{itemize}
\item Time dependent: allow double amount of timeout to react, but accept any reaction after the timeout period.
\item Generating wrong keyshare or confirmation share:
\begin{itemize}
\item Generate a new share (not to disclose the previous secret).
\item In case of failure, encrypt the new shares using AES-GCM with the old p2p key and send it to everybody.
\end{itemize}
\end{itemize}
\end{itemize}

\section{How to react after detection}
\label{sec-5}
\begin{itemize}
\item If DoS happens during a join process, 
If it is the joiner who is malicious:
   The maliciousness is happening in session confirmation phase:
\begin{itemize}
\item just drop the session in limbo. Send re-join message with participant info without DoSer.
\item The maliciousness is happening before sending session confirmation phase. Just drop the joining session.
\end{itemize}
A current participant is malicious:
\begin{itemize}
\item send a kick request which generates a session confirmation that helps the joiner to know that a new session is generated.
\end{itemize}

\item If DoS happens during leave process:
Send a kick request for the malicious participant.

\item If DoS happens during re-session.
Send a kick request for the malicious participant.

\item Authentication failure is a reason for barring join but not DoS.

\item When someone gets kicked out due to DoS reasons, she should become the 
last person to join after all the other joiners already in line.
\end{itemize}

\section{Concerns:}
\label{sec-6}
\begin{itemize}
\item Timing problems. There should be an acceptable delay. The messages arrived within an
\end{itemize}
acceptable delay period should be ordered in their hash order. But we 
assume global ordering on messages for now.

\begin{itemize}
\item Multiple sessions in the same room. This is a natural consequence of the end-to-end protocol, as a different subgroup might have agreed on different views. This is not a problem with current participants, as they are ignoring the session id for which they do not have an established session.
\item For a joining participant, the UI will present a choice of sessions and participants which the joiner can choose to join.
\item For the sake of simplicity, we assume that each room is divided into mutually exclusive sessions.
\end{itemize}

\section{Proof of DoS protection}
\label{sec-7}
\begin{itemize}
\item Theorem: Suppose $U_1,...,U_n$ are sets of participants. $I_h \cap I_m = \{1,...,n\}$.
\end{itemize}
where $I_o$ is the set of honest and $I_m$ the set of malcious participants, then, after running the above algorithm, each participant gets a list of $plist_i$. If the transport is honestly and consistently delivering messages in timely manner, then for $i,j \in I_h$  
we have $U_i \in plist_j$.

Proof: TBD.

\section{New Algorithm:}
\label{sec-8}
\begin{itemize}
\item Badly signed messages are dropped and treated as undelivered.
\item If someone fails to contribute any message we are waiting for, we wait for the grace$_{\text{period}}$
    and then we just assume they left.
\item If key generation or confirmation fails, then we need to re-session with the session 
tagged as DoS detection. Users only can do this by sharing the evidence of cheating.
\item If we fail key generation or confirmation with Dos detection tag, then we publish
all private key encrypted by p2p keys signed by non-DoS-tagged authenticated
private key. The cheater will be detected and kicked out.
\item If U$_{\text{i}}$ kicks U$_{\text{j}}$ out (that is starting session S while U$_{\text{j}}$ is not in the 
new session) without cheating evidence signed by alleged cheater U$_{\text{j}}$, then U$_{\text{k}}$ simply ignore the request 
for the new session.
\end{itemize}

\subsection{Sub protocol for unresponsiveness on join or re-session or any other part of the key agreement protocol.}
\label{sec-8-1}
If U$_{\text{i}}$ fails to reply, U$_{\text{j}}$ sends a kick request (for failure to reply) after the grace period has passed.
Other participants either should agree with the kick and respond by participant-info message or re-broadcast the failed message (if they have received it despite the fact that U$_{\text{j}}$ has not). When 
you get the message for failed delivery, you \textbf{have to} agree or rebroadcast. If you fail to do so, you will be dropped as well.

\begin{itemize}
\item The main idea is to have two timers:
A replies before timer 1 ends, B asks to kick out A=> kickout B.
A replies after timer 1 but before timer 2 ends, B asks to kick out A => kickout A.
A replies after timer 2 but B does not ask to kickout A => ask to kickout A.
\end{itemize}

-- Current users start a timer as soon as they get a join request for 
all users in the room to respond with authentication.

Note: Authentication failed should be an acceptable response.

-- If the timer times out, they mark the user as unresponsive and start
another timer to report the user as unresponsive.

-- If they receive the message before the timer times out,
   and no other user requests the user to leave, they continue with
   the session establishment.

-- If somebody requests a new session without the user, they accept
   the request and can drop the session.

-- If they don't receive the message before the second timer times
   out, they request a session without the unresponsive users.

-- If they receive a session shrink but the kicked-out user
   has replied in time for the first timer, they drop the requesting
   user. (We should because the receiving user and the requesting user have different views of the room.)

The users will play the second round, set up timers and follow the same 
rules.

-- Conflict of circles:
   Obviously there will be a circle conflict because:

A thinks B is in and C is not. (A,B) => A eventually doesn't receive a response from B and drops B.
B thinks C is in and A is not. (B,C) =>  B doesn't receive a response from C and drops C.
C thinks A, B and C are in. (A,B,C) => C doesn't receive a response from A and B and drops both of them.

\begin{itemize}
\item In the end only users whose views are completely in agreement will stay in the same session. The room might be
divided into different mutually exclusive sessions. (View agreement is an equivalence relationship.) 
The new joiner will be presented the option of choosing which session to join.
\end{itemize}

So it is obvious that we should treat cheating and transport problems separately, because the latter are provable.

in particular, if A decides that C is unresponsive while C is responsive, then B can relay C info signed by C to A.

When B receives a DoS re-session request from A to drop C: either B agrees with A on the reason of dropping C or she doesn't. If yes, she continues with the kick protocol on C. If no, B re-send the missing info to A in hope of reaching agreement if A doesn't reply to requested info in timely manner, then B's view does not agree with A's view, and therefore they cannot be in the same session. As such, B sends a kick request for A and all other participants whose view agrees with B drop A and start a new session.

\subsubsection{Delivery failure/Transport delay recovery.}
\label{sec-8-1-1}
\begin{itemize}
\item The protocol agrees on INTERACTIVE$_{\text{GRACE}}$$_{\text{PERIOD}}$.
\item If U$_{\text{i}}$ expects a message from U$_{\text{j}}$ to establish a session and she does not receive it in INTERACTIVE$_{\text{GRACE}}$$_{\text{PERIOD}}$ as of the end of the last round, then U$_{\text{i}}$ starts new$_{\text{session}}$(kick$_{\text{out}}$ U$_{\text{i}}$, reason: U$_{\text{i}}$ fails message type x from session sid).
\item If U$_{\text{k}}$ receives a kick message, they either have received the failed delivery message or they have not: if they have, they resend the message to s'id, message type x from sid by U$_{\text{j}}$ (encrypt or not?) and do not follow up with the new
session.
\item If U$_{\text{i}}$ is the only member of the session and there are more participants in the room, U$_{\text{i}}$ will rejoin the room.
\end{itemize}

\subsection{Sub protocol for generating wrong keyshare or confirming wrong session.}
\label{sec-8-2}

The protocol general rule:

-- If key recovery or confirmation fails, then the key agreement protocol runs again with new ephemeral keys. If it fails, then the new keys and proof of cheating are published. A new session starts when the cheater has been kicked out.

-- If someone fails to reply, then run sub-protocol for unresponsive user.

The failed message has the share for the new subgroup, so finally the responsive 
parties will make a successful subgroup.

So we break the protocol into sections:

\subsubsection{Cheater detection protocol:}
\label{sec-8-2-1}

   If the key fails to recover or the session confirmation does not match, then a special session with new ephemeral keys will be distributed and session establishment will be attempted. If the cheater detection session fails 
at the same stages, then the participants will reveal their private key signed by their old key and the cheater will be detected and kicked out.

\begin{itemize}
\item U$_{\text{i}}$ fails at key recovery or conf$_{\text{j}}$!= conf$_{\text{i}}$. Request a cheater detection session, with reason. (Reason is the set of signed shares which does not satisfy the system or the confirmation which does not match the deduced key.)
\item U$_{\text{j}}$ receives a request for cheater detection and evaluates the reason. If it is legitimate, a cheater detection session starts.
\item If the cheater detection succeeds, it becomes the main session.
\item If the cheater detection session fails, private keys signed by old private keys are published.
\item Detect cheater and kick them out with proof.
\item Do not join sessions which you do not agree with on the view.
\end{itemize}
% Emacs 24.5.1 (Org mode 8.2.10)
\end{document}
